\documentclass[]{article}
\usepackage{lmodern}
\usepackage{amssymb,amsmath}
\usepackage{ifxetex,ifluatex}
\usepackage{fixltx2e} % provides \textsubscript
\ifnum 0\ifxetex 1\fi\ifluatex 1\fi=0 % if pdftex
  \usepackage[T1]{fontenc}
  \usepackage[utf8]{inputenc}
\else % if luatex or xelatex
  \ifxetex
    \usepackage{mathspec}
  \else
    \usepackage{fontspec}
  \fi
  \defaultfontfeatures{Ligatures=TeX,Scale=MatchLowercase}
\fi
% use upquote if available, for straight quotes in verbatim environments
\IfFileExists{upquote.sty}{\usepackage{upquote}}{}
% use microtype if available
\IfFileExists{microtype.sty}{%
\usepackage{microtype}
\UseMicrotypeSet[protrusion]{basicmath} % disable protrusion for tt fonts
}{}
\usepackage[margin=1in]{geometry}
\usepackage{hyperref}
\hypersetup{unicode=true,
            pdftitle={results reproduction},
            pdfauthor={Jocelyn Hu},
            pdfborder={0 0 0},
            breaklinks=true}
\urlstyle{same}  % don't use monospace font for urls
\usepackage{color}
\usepackage{fancyvrb}
\newcommand{\VerbBar}{|}
\newcommand{\VERB}{\Verb[commandchars=\\\{\}]}
\DefineVerbatimEnvironment{Highlighting}{Verbatim}{commandchars=\\\{\}}
% Add ',fontsize=\small' for more characters per line
\usepackage{framed}
\definecolor{shadecolor}{RGB}{248,248,248}
\newenvironment{Shaded}{\begin{snugshade}}{\end{snugshade}}
\newcommand{\KeywordTok}[1]{\textcolor[rgb]{0.13,0.29,0.53}{\textbf{#1}}}
\newcommand{\DataTypeTok}[1]{\textcolor[rgb]{0.13,0.29,0.53}{#1}}
\newcommand{\DecValTok}[1]{\textcolor[rgb]{0.00,0.00,0.81}{#1}}
\newcommand{\BaseNTok}[1]{\textcolor[rgb]{0.00,0.00,0.81}{#1}}
\newcommand{\FloatTok}[1]{\textcolor[rgb]{0.00,0.00,0.81}{#1}}
\newcommand{\ConstantTok}[1]{\textcolor[rgb]{0.00,0.00,0.00}{#1}}
\newcommand{\CharTok}[1]{\textcolor[rgb]{0.31,0.60,0.02}{#1}}
\newcommand{\SpecialCharTok}[1]{\textcolor[rgb]{0.00,0.00,0.00}{#1}}
\newcommand{\StringTok}[1]{\textcolor[rgb]{0.31,0.60,0.02}{#1}}
\newcommand{\VerbatimStringTok}[1]{\textcolor[rgb]{0.31,0.60,0.02}{#1}}
\newcommand{\SpecialStringTok}[1]{\textcolor[rgb]{0.31,0.60,0.02}{#1}}
\newcommand{\ImportTok}[1]{#1}
\newcommand{\CommentTok}[1]{\textcolor[rgb]{0.56,0.35,0.01}{\textit{#1}}}
\newcommand{\DocumentationTok}[1]{\textcolor[rgb]{0.56,0.35,0.01}{\textbf{\textit{#1}}}}
\newcommand{\AnnotationTok}[1]{\textcolor[rgb]{0.56,0.35,0.01}{\textbf{\textit{#1}}}}
\newcommand{\CommentVarTok}[1]{\textcolor[rgb]{0.56,0.35,0.01}{\textbf{\textit{#1}}}}
\newcommand{\OtherTok}[1]{\textcolor[rgb]{0.56,0.35,0.01}{#1}}
\newcommand{\FunctionTok}[1]{\textcolor[rgb]{0.00,0.00,0.00}{#1}}
\newcommand{\VariableTok}[1]{\textcolor[rgb]{0.00,0.00,0.00}{#1}}
\newcommand{\ControlFlowTok}[1]{\textcolor[rgb]{0.13,0.29,0.53}{\textbf{#1}}}
\newcommand{\OperatorTok}[1]{\textcolor[rgb]{0.81,0.36,0.00}{\textbf{#1}}}
\newcommand{\BuiltInTok}[1]{#1}
\newcommand{\ExtensionTok}[1]{#1}
\newcommand{\PreprocessorTok}[1]{\textcolor[rgb]{0.56,0.35,0.01}{\textit{#1}}}
\newcommand{\AttributeTok}[1]{\textcolor[rgb]{0.77,0.63,0.00}{#1}}
\newcommand{\RegionMarkerTok}[1]{#1}
\newcommand{\InformationTok}[1]{\textcolor[rgb]{0.56,0.35,0.01}{\textbf{\textit{#1}}}}
\newcommand{\WarningTok}[1]{\textcolor[rgb]{0.56,0.35,0.01}{\textbf{\textit{#1}}}}
\newcommand{\AlertTok}[1]{\textcolor[rgb]{0.94,0.16,0.16}{#1}}
\newcommand{\ErrorTok}[1]{\textcolor[rgb]{0.64,0.00,0.00}{\textbf{#1}}}
\newcommand{\NormalTok}[1]{#1}
\usepackage{graphicx,grffile}
\makeatletter
\def\maxwidth{\ifdim\Gin@nat@width>\linewidth\linewidth\else\Gin@nat@width\fi}
\def\maxheight{\ifdim\Gin@nat@height>\textheight\textheight\else\Gin@nat@height\fi}
\makeatother
% Scale images if necessary, so that they will not overflow the page
% margins by default, and it is still possible to overwrite the defaults
% using explicit options in \includegraphics[width, height, ...]{}
\setkeys{Gin}{width=\maxwidth,height=\maxheight,keepaspectratio}
\IfFileExists{parskip.sty}{%
\usepackage{parskip}
}{% else
\setlength{\parindent}{0pt}
\setlength{\parskip}{6pt plus 2pt minus 1pt}
}
\setlength{\emergencystretch}{3em}  % prevent overfull lines
\providecommand{\tightlist}{%
  \setlength{\itemsep}{0pt}\setlength{\parskip}{0pt}}
\setcounter{secnumdepth}{0}
% Redefines (sub)paragraphs to behave more like sections
\ifx\paragraph\undefined\else
\let\oldparagraph\paragraph
\renewcommand{\paragraph}[1]{\oldparagraph{#1}\mbox{}}
\fi
\ifx\subparagraph\undefined\else
\let\oldsubparagraph\subparagraph
\renewcommand{\subparagraph}[1]{\oldsubparagraph{#1}\mbox{}}
\fi

%%% Use protect on footnotes to avoid problems with footnotes in titles
\let\rmarkdownfootnote\footnote%
\def\footnote{\protect\rmarkdownfootnote}

%%% Change title format to be more compact
\usepackage{titling}

% Create subtitle command for use in maketitle
\providecommand{\subtitle}[1]{
  \posttitle{
    \begin{center}\large#1\end{center}
    }
}

\setlength{\droptitle}{-2em}

  \title{results reproduction}
    \pretitle{\vspace{\droptitle}\centering\huge}
  \posttitle{\par}
    \author{Jocelyn Hu}
    \preauthor{\centering\large\emph}
  \postauthor{\par}
    \date{}
    \predate{}\postdate{}
  
\usepackage{booktabs}
\usepackage{longtable}
\usepackage{array}
\usepackage{multirow}
\usepackage{wrapfig}
\usepackage{float}
\usepackage{colortbl}
\usepackage{pdflscape}
\usepackage{tabu}
\usepackage{threeparttable}
\usepackage{threeparttablex}
\usepackage[normalem]{ulem}
\usepackage{makecell}
\usepackage{xcolor}

\begin{document}
\maketitle

\begin{Shaded}
\begin{Highlighting}[]
\KeywordTok{library}\NormalTok{(tidyverse)}
\end{Highlighting}
\end{Shaded}

\begin{verbatim}
## -- Attaching packages -------------- tidyverse 1.2.1 --
\end{verbatim}

\begin{verbatim}
## v ggplot2 3.1.1       v purrr   0.3.2  
## v tibble  2.1.1       v dplyr   0.8.0.1
## v tidyr   0.8.3       v stringr 1.4.0  
## v readr   1.3.1       v forcats 0.4.0
\end{verbatim}

\begin{verbatim}
## -- Conflicts ----------------- tidyverse_conflicts() --
## x dplyr::filter() masks stats::filter()
## x dplyr::lag()    masks stats::lag()
\end{verbatim}

\begin{Shaded}
\begin{Highlighting}[]
\KeywordTok{library}\NormalTok{(yardstick)}
\end{Highlighting}
\end{Shaded}

\begin{verbatim}
## For binary classification, the first factor level is assumed to be the event.
## Set the global option `yardstick.event_first` to `FALSE` to change this.
\end{verbatim}

\begin{verbatim}
## 
## Attaching package: 'yardstick'
\end{verbatim}

\begin{verbatim}
## The following object is masked from 'package:readr':
## 
##     spec
\end{verbatim}

\begin{Shaded}
\begin{Highlighting}[]
\KeywordTok{library}\NormalTok{(ggplot2)}
\KeywordTok{library}\NormalTok{(mosaic)}
\end{Highlighting}
\end{Shaded}

\begin{verbatim}
## Loading required package: lattice
\end{verbatim}

\begin{verbatim}
## Loading required package: ggformula
\end{verbatim}

\begin{verbatim}
## Loading required package: ggstance
\end{verbatim}

\begin{verbatim}
## 
## Attaching package: 'ggstance'
\end{verbatim}

\begin{verbatim}
## The following objects are masked from 'package:ggplot2':
## 
##     geom_errorbarh, GeomErrorbarh
\end{verbatim}

\begin{verbatim}
## 
## New to ggformula?  Try the tutorials: 
##  learnr::run_tutorial("introduction", package = "ggformula")
##  learnr::run_tutorial("refining", package = "ggformula")
\end{verbatim}

\begin{verbatim}
## Loading required package: mosaicData
\end{verbatim}

\begin{verbatim}
## Loading required package: Matrix
\end{verbatim}

\begin{verbatim}
## 
## Attaching package: 'Matrix'
\end{verbatim}

\begin{verbatim}
## The following object is masked from 'package:tidyr':
## 
##     expand
\end{verbatim}

\begin{verbatim}
## 
## The 'mosaic' package masks several functions from core packages in order to add 
## additional features.  The original behavior of these functions should not be affected by this.
## 
## Note: If you use the Matrix package, be sure to load it BEFORE loading mosaic.
\end{verbatim}

\begin{verbatim}
## 
## Attaching package: 'mosaic'
\end{verbatim}

\begin{verbatim}
## The following object is masked from 'package:Matrix':
## 
##     mean
\end{verbatim}

\begin{verbatim}
## The following objects are masked from 'package:dplyr':
## 
##     count, do, tally
\end{verbatim}

\begin{verbatim}
## The following object is masked from 'package:purrr':
## 
##     cross
\end{verbatim}

\begin{verbatim}
## The following object is masked from 'package:ggplot2':
## 
##     stat
\end{verbatim}

\begin{verbatim}
## The following objects are masked from 'package:stats':
## 
##     binom.test, cor, cor.test, cov, fivenum, IQR, median,
##     prop.test, quantile, sd, t.test, var
\end{verbatim}

\begin{verbatim}
## The following objects are masked from 'package:base':
## 
##     max, mean, min, prod, range, sample, sum
\end{verbatim}

\begin{Shaded}
\begin{Highlighting}[]
\KeywordTok{library}\NormalTok{(stargazer)}
\end{Highlighting}
\end{Shaded}

\begin{verbatim}
## 
## Please cite as:
\end{verbatim}

\begin{verbatim}
##  Hlavac, Marek (2018). stargazer: Well-Formatted Regression and Summary Statistics Tables.
\end{verbatim}

\begin{verbatim}
##  R package version 5.2.2. https://CRAN.R-project.org/package=stargazer
\end{verbatim}

\begin{Shaded}
\begin{Highlighting}[]
\KeywordTok{require}\NormalTok{(Stat2Data)}
\end{Highlighting}
\end{Shaded}

\begin{verbatim}
## Loading required package: Stat2Data
\end{verbatim}

\begin{Shaded}
\begin{Highlighting}[]
\KeywordTok{library}\NormalTok{(scales)}
\end{Highlighting}
\end{Shaded}

\begin{verbatim}
## 
## Attaching package: 'scales'
\end{verbatim}

\begin{verbatim}
## The following object is masked from 'package:mosaic':
## 
##     rescale
\end{verbatim}

\begin{verbatim}
## The following object is masked from 'package:purrr':
## 
##     discard
\end{verbatim}

\begin{verbatim}
## The following object is masked from 'package:readr':
## 
##     col_factor
\end{verbatim}

\begin{Shaded}
\begin{Highlighting}[]
\KeywordTok{library}\NormalTok{(readr)}
\KeywordTok{library}\NormalTok{(rms)}
\end{Highlighting}
\end{Shaded}

\begin{verbatim}
## Loading required package: Hmisc
\end{verbatim}

\begin{verbatim}
## Loading required package: survival
\end{verbatim}

\begin{verbatim}
## Loading required package: Formula
\end{verbatim}

\begin{verbatim}
## 
## Attaching package: 'Hmisc'
\end{verbatim}

\begin{verbatim}
## The following objects are masked from 'package:dplyr':
## 
##     src, summarize
\end{verbatim}

\begin{verbatim}
## The following objects are masked from 'package:base':
## 
##     format.pval, units
\end{verbatim}

\begin{verbatim}
## Loading required package: SparseM
\end{verbatim}

\begin{verbatim}
## 
## Attaching package: 'SparseM'
\end{verbatim}

\begin{verbatim}
## The following object is masked from 'package:base':
## 
##     backsolve
\end{verbatim}

\begin{Shaded}
\begin{Highlighting}[]
\KeywordTok{library}\NormalTok{(haven)}
\NormalTok{Dyadic_dta<-}\StringTok{ }\KeywordTok{read_dta}\NormalTok{(}\StringTok{"Data/AttanasioEtAl2011Dyadic.dta"}\NormalTok{)}
\NormalTok{Dyadic<-}\KeywordTok{read.csv}\NormalTok{(}\StringTok{"Dyadic.csv"}\NormalTok{)}
\NormalTok{Dyadic_mdum<-}\KeywordTok{read.csv}\NormalTok{(}\StringTok{"Dyadic_mdum.csv"}\NormalTok{)}
\end{Highlighting}
\end{Shaded}

\begin{Shaded}
\begin{Highlighting}[]
\KeywordTok{glimpse}\NormalTok{(Dyadic_dta}\OperatorTok{$}\NormalTok{inreg)}
\end{Highlighting}
\end{Shaded}

\begin{Shaded}
\begin{Highlighting}[]
\KeywordTok{str}\NormalTok{(Dyadic)}
\end{Highlighting}
\end{Shaded}

Mutate var stranger which does no exist in the original dataframe

\begin{Shaded}
\begin{Highlighting}[]
\KeywordTok{tally}\NormalTok{(}\OperatorTok{~}\NormalTok{Dyadic}\OperatorTok{$}\NormalTok{municode)}
\end{Highlighting}
\end{Shaded}

\begin{verbatim}
## Dyadic$municode
##    1    2    3    4    5    6    7    8    9   10   11   12   13   14   15 
## 1122  702  342  210  756 1560  462  992  462  420 2162  650 1122 1260 1122 
##   16   17   18   19   20   21   22   23   24   25   26   27   28   29   30 
##  600  812 1332 2256 1806  702  272 1332  756  210  992 1482  756 1406  650 
##   31   32   33   34   35   36   37   38   39   40   41   42   43   44   45 
## 1806  992 1640  600  930  240 1560 1892 2756 2070 1332 2352 1560  182   72 
##   46   47   48   49   50   51   52   53   54   55   56   57   58   59   60 
##  992  342 1332  992  462 1560 1406 2550 2162 1406  992  992  462 1722 1482 
##   61   62   63   64   65   66   67   68   69   70 
## 2256 1640 1406 1722 2450 1332 7482  506  870 1056
\end{verbatim}

\begin{Shaded}
\begin{Highlighting}[]
\KeywordTok{tally}\NormalTok{(}\OperatorTok{~}\NormalTok{Dyadic}\OperatorTok{$}\NormalTok{samegroup)}
\end{Highlighting}
\end{Shaded}

\begin{verbatim}
## Dyadic$samegroup
##     0     1 
## 78892  9374
\end{verbatim}

\begin{Shaded}
\begin{Highlighting}[]
\CommentTok{#Dyadic<-na.omit(Dyadic) }
\NormalTok{Dyadic<-Dyadic}\OperatorTok
\StringTok{  }\KeywordTok{mutate}\NormalTok{(}\DataTypeTok{stranger =} \DecValTok{1} \OperatorTok{-}\StringTok{ }\KeywordTok{as.numeric}\NormalTok{(Dyadic}\OperatorTok{$}\NormalTok{frfam))}
\end{Highlighting}
\end{Shaded}

\begin{Shaded}
\begin{Highlighting}[]
\NormalTok{star<-Dyadic}
\NormalTok{pervar<-}\KeywordTok{c}\NormalTok{(}\StringTok{"samegroup"}\NormalTok{,}\StringTok{"friend2"}\NormalTok{,}\StringTok{"friendfamily"}\NormalTok{,}\StringTok{"stranger"}\NormalTok{, }\StringTok{"frfam"}\NormalTok{,}\StringTok{"family2"}\NormalTok{, }\StringTok{"friend1"}\NormalTok{, }\StringTok{"family1"}\NormalTok{, }\StringTok{"frfamcl"}\NormalTok{, }\StringTok{"difurban"}\NormalTok{, }\StringTok{"diffemale"}\NormalTok{,}\StringTok{"difmarried"}\NormalTok{)}
\ControlFlowTok{for}\NormalTok{ (i }\ControlFlowTok{in}\NormalTok{ pervar) \{}
\NormalTok{star[[i]] <-}\StringTok{ }\NormalTok{(}\KeywordTok{as.numeric}\NormalTok{(Dyadic[[i]]))}\OperatorTok{*}\DecValTok{100}
\NormalTok{\}}
\CommentTok{# }
\CommentTok{# Dyadic[["sumtcons"]] <- (as.numeric(Dyadic[["sumtcons"]]))/1000}
\CommentTok{# }
\CommentTok{# Dyadic[["diftcons"]] <- (as.numeric(Dyadic[["diftcons"]]))/1000}
\end{Highlighting}
\end{Shaded}

\url{https://www.princeton.edu/~otorres/NiceOutputR.pdf}

The variable difmar's name should be `difmarried' and nummar should be
nummarried.

\begin{Shaded}
\begin{Highlighting}[]
\KeywordTok{stargazer}\NormalTok{(}\KeywordTok{subset}\NormalTok{(star[}\KeywordTok{c}\NormalTok{(}\StringTok{"samegroup"}\NormalTok{,}\StringTok{"difchoice1"}\NormalTok{,}\StringTok{"sumchoice1"}\NormalTok{,}\StringTok{"frfam"}\NormalTok{,}\StringTok{"friend2"}\NormalTok{,}\StringTok{"friendfamily"}\NormalTok{, }\StringTok{"family2"}\NormalTok{, }\StringTok{"friend1"}\NormalTok{, }\StringTok{"family1"}\NormalTok{,}\StringTok{"stranger"}\NormalTok{,}\StringTok{"frfamcl"}\NormalTok{, }\StringTok{"difurban"}\NormalTok{, }\StringTok{"diffemale"}\NormalTok{, }\StringTok{"difyage"}\NormalTok{, }\StringTok{"difysch"}\NormalTok{, }\StringTok{"difmarried"}\NormalTok{,}\StringTok{"diftcons"}\NormalTok{,}\StringTok{"difcons"}\NormalTok{, }\StringTok{"difhhsize"}\NormalTok{,}\StringTok{"difwin1"}\NormalTok{,}\StringTok{"sumurban"}\NormalTok{,}\StringTok{"sumfemale"}\NormalTok{,}\StringTok{"sumyage"}\NormalTok{, }\StringTok{"sumysch"}\NormalTok{, }\StringTok{"nummarried"}\NormalTok{, }\StringTok{"sumtcons"}\NormalTok{, }\StringTok{"sumcons"}\NormalTok{, }\StringTok{"sumhhsize"}\NormalTok{, }\StringTok{"sumwin1"}\NormalTok{)],star}\OperatorTok{$}\NormalTok{inreg }\OperatorTok{==}\DecValTok{1}\NormalTok{), }\DataTypeTok{type =} \StringTok{"html"}\NormalTok{, }\DataTypeTok{title=}\StringTok{"Table4: Dyadic Variables"}\NormalTok{, }\DataTypeTok{digits=}\DecValTok{4}\NormalTok{, }\DataTypeTok{out=}\StringTok{"table4.html"}\NormalTok{, }\DataTypeTok{covariate.labels =} \KeywordTok{c}\NormalTok{(}\StringTok{"Joined same group in round 2(%)"}\NormalTok{, }\StringTok{"Difference in gamble choice(round1)"}\NormalTok{,}\StringTok{"Sum of gamble choices (round 1)"}\NormalTok{, }\StringTok{"Friends and family: One or both recognized friendship or kinship(%)"}\NormalTok{,}\StringTok{"Both recognized friendship(%)"}\NormalTok{,}\StringTok{"Both recognized kinship(%)"}\NormalTok{,}\StringTok{"One recognized friendship, other kinship(%)"}\NormalTok{,}\StringTok{"One recognized friendship(%)"}\NormalTok{, }\StringTok{"One recognized kinship(%)"}\NormalTok{, }\StringTok{"Stranger(%)"}\NormalTok{,}\StringTok{"Geographically proximate friends and family(%)"}\NormalTok{,}\StringTok{"One lives in the municipal centre, one not(%)"}\NormalTok{,}\StringTok{"Different genders(%)"}\NormalTok{,}\StringTok{"Difference in age(years)"}\NormalTok{,}\StringTok{"Difference in education (years)"}\NormalTok{,}\StringTok{"Difference in marital status(%)"}\NormalTok{,}\StringTok{"Difference in household consumption ('000s Pesos/month)"}\NormalTok{,}\StringTok{"Difference in log household consumption per month"}\NormalTok{,}\StringTok{"Difference in household size"}\NormalTok{,}\StringTok{"Difference in round 1 winnings ('000 Pesos)"}\NormalTok{,}\StringTok{"Number who live in the municipal centre"}\NormalTok{,}\StringTok{"Number of females"}\NormalTok{,}\StringTok{"Sum of ages (years)"}\NormalTok{,}\StringTok{"Sum of education (years)"}\NormalTok{,}\StringTok{"Number married"}\NormalTok{,}\StringTok{"Sum of household consumption ('000s Pesos/month)"}\NormalTok{,}\StringTok{"Sum of log household consumption per month"}\NormalTok{,}\StringTok{"Sum of household sizes"}\NormalTok{,}\StringTok{"Sum of round 1 winnings ('000s Pesos)"}\NormalTok{), }\DataTypeTok{summary.stat =} \KeywordTok{c}\NormalTok{(}\StringTok{"mean"}\NormalTok{, }\StringTok{"sd"}\NormalTok{))}
\end{Highlighting}
\end{Shaded}

\begin{verbatim}
## 
## <table style="text-align:center"><caption><strong>Table4: Dyadic Variables</strong></caption>
## <tr><td colspan="3" style="border-bottom: 1px solid black"></td></tr><tr><td style="text-align:left">Statistic</td><td>Mean</td><td>St. Dev.</td></tr>
## <tr><td colspan="3" style="border-bottom: 1px solid black"></td></tr><tr><td style="text-align:left">Joined same group in round 2(%)</td><td>9.2096</td><td>28.9164</td></tr>
## <tr><td style="text-align:left">Difference in gamble choice(round1)</td><td>1.6391</td><td>1.2648</td></tr>
## <tr><td style="text-align:left">Sum of gamble choices (round 1)</td><td>7.1789</td><td>2.1228</td></tr>
## <tr><td style="text-align:left">Friends and family: One or both recognized friendship or kinship(%)</td><td>10.4903</td><td>30.6430</td></tr>
## <tr><td style="text-align:left">Both recognized friendship(%)</td><td>2.4272</td><td>15.3895</td></tr>
## <tr><td style="text-align:left">Both recognized kinship(%)</td><td>0.1757</td><td>4.1878</td></tr>
## <tr><td style="text-align:left">One recognized friendship, other kinship(%)</td><td>0.4531</td><td>6.7159</td></tr>
## <tr><td style="text-align:left">One recognized friendship(%)</td><td>6.9003</td><td>25.3461</td></tr>
## <tr><td style="text-align:left">One recognized kinship(%)</td><td>0.5340</td><td>7.2880</td></tr>
## <tr><td style="text-align:left">Stranger(%)</td><td>89.5097</td><td>30.6430</td></tr>
## <tr><td style="text-align:left">Geographically proximate friends and family(%)</td><td>5.1619</td><td>22.1259</td></tr>
## <tr><td style="text-align:left">One lives in the municipal centre, one not(%)</td><td>30.9508</td><td>46.2293</td></tr>
## <tr><td style="text-align:left">Different genders(%)</td><td>20.5368</td><td>40.3972</td></tr>
## <tr><td style="text-align:left">Difference in age(years)</td><td>12.4041</td><td>9.6816</td></tr>
## <tr><td style="text-align:left">Difference in education (years)</td><td>3.2346</td><td>2.7703</td></tr>
## <tr><td style="text-align:left">Difference in marital status(%)</td><td>34.6841</td><td>47.5967</td></tr>
## <tr><td style="text-align:left">Difference in household consumption ('000s Pesos/month)</td><td>232,840.3000</td><td>227,243.5000</td></tr>
## <tr><td style="text-align:left">Difference in log household consumption per month</td><td>0.5894</td><td>0.4892</td></tr>
## <tr><td style="text-align:left">Difference in household size</td><td>3.1112</td><td>2.9074</td></tr>
## <tr><td style="text-align:left">Difference in round 1 winnings ('000 Pesos)</td><td>4.1820</td><td>3.2132</td></tr>
## <tr><td style="text-align:left">Number who live in the municipal centre</td><td>0.7148</td><td>0.7805</td></tr>
## <tr><td style="text-align:left">Number of females</td><td>1.7501</td><td>0.4816</td></tr>
## <tr><td style="text-align:left">Sum of ages (years)</td><td>83.6733</td><td>16.0116</td></tr>
## <tr><td style="text-align:left">Sum of education (years)</td><td>7.3521</td><td>4.5120</td></tr>
## <tr><td style="text-align:left">Number married</td><td>1.5503</td><td>0.5919</td></tr>
## <tr><td style="text-align:left">Sum of household consumption ('000s Pesos/month)</td><td>850,188.2000</td><td>359,941.9000</td></tr>
## <tr><td style="text-align:left">Sum of log household consumption per month</td><td>25.6207</td><td>0.8454</td></tr>
## <tr><td style="text-align:left">Sum of household sizes</td><td>14.5676</td><td>4.5292</td></tr>
## <tr><td style="text-align:left">Sum of round 1 winnings ('000s Pesos)</td><td>11.7083</td><td>5.4843</td></tr>
## <tr><td colspan="3" style="border-bottom: 1px solid black"></td></tr></table>
\end{verbatim}

\begin{Shaded}
\begin{Highlighting}[]
\KeywordTok{library}\NormalTok{(dplyr)}
\NormalTok{Dyadic1 <-}\StringTok{ }\NormalTok{Dyadic }\OperatorTok
\StringTok{  }\KeywordTok{subset}\NormalTok{(inreg }\OperatorTok{==}\StringTok{ }\DecValTok{1}\NormalTok{)}\OperatorTok
\StringTok{  }\KeywordTok{mutate}\NormalTok{(}\DataTypeTok{Observations =} \DecValTok{0}\NormalTok{)}\OperatorTok
\StringTok{  }\KeywordTok{select}\NormalTok{(}\KeywordTok{c}\NormalTok{(}\StringTok{"samegroup"}\NormalTok{,}\StringTok{"difchoice1"}\NormalTok{,}\StringTok{"sumchoice1"}\NormalTok{,}\StringTok{"frfam"}\NormalTok{,}\StringTok{"friend2"}\NormalTok{,}\StringTok{"friendfamily"}\NormalTok{, }\StringTok{"family2"}\NormalTok{, }\StringTok{"friend1"}\NormalTok{, }\StringTok{"family1"}\NormalTok{,}\StringTok{"stranger"}\NormalTok{,}\StringTok{"frfamcl"}\NormalTok{, }\StringTok{"difurban"}\NormalTok{, }\StringTok{"diffemale"}\NormalTok{, }\StringTok{"difyage"}\NormalTok{, }\StringTok{"difysch"}\NormalTok{, }\StringTok{"difmarried"}\NormalTok{,}\StringTok{"diftcons"}\NormalTok{,}\StringTok{"difcons"}\NormalTok{, }\StringTok{"difhhsize"}\NormalTok{,}\StringTok{"difwin1"}\NormalTok{,}\StringTok{"sumurban"}\NormalTok{,}\StringTok{"sumfemale"}\NormalTok{,}\StringTok{"sumyage"}\NormalTok{, }\StringTok{"sumysch"}\NormalTok{, }\StringTok{"nummarried"}\NormalTok{, }\StringTok{"sumtcons"}\NormalTok{, }\StringTok{"sumcons"}\NormalTok{, }\StringTok{"sumhhsize"}\NormalTok{, }\StringTok{"sumwin1"}\NormalTok{,}\StringTok{"Observations"}\NormalTok{))}


\NormalTok{Dyadic2 <-}\StringTok{ }\NormalTok{Dyadic }\OperatorTok
\StringTok{  }\KeywordTok{subset}\NormalTok{(inreg }\OperatorTok{==}\StringTok{ }\DecValTok{1} \OperatorTok{&}\StringTok{ }\NormalTok{frfamcl}\OperatorTok{==}\StringTok{ }\DecValTok{1}\NormalTok{)}\OperatorTok
\StringTok{  }\KeywordTok{mutate}\NormalTok{(}\DataTypeTok{Observations =} \DecValTok{0}\NormalTok{)}\OperatorTok
\StringTok{  }\KeywordTok{select}\NormalTok{(}\KeywordTok{c}\NormalTok{(}\StringTok{"samegroup"}\NormalTok{,}\StringTok{"difchoice1"}\NormalTok{,}\StringTok{"sumchoice1"}\NormalTok{,}\StringTok{"frfam"}\NormalTok{,}\StringTok{"friend2"}\NormalTok{,}\StringTok{"friendfamily"}\NormalTok{, }\StringTok{"family2"}\NormalTok{, }\StringTok{"friend1"}\NormalTok{, }\StringTok{"family1"}\NormalTok{,}\StringTok{"stranger"}\NormalTok{,}\StringTok{"frfamcl"}\NormalTok{, }\StringTok{"difurban"}\NormalTok{, }\StringTok{"diffemale"}\NormalTok{, }\StringTok{"difyage"}\NormalTok{, }\StringTok{"difysch"}\NormalTok{, }\StringTok{"difmarried"}\NormalTok{,}\StringTok{"diftcons"}\NormalTok{,}\StringTok{"difcons"}\NormalTok{, }\StringTok{"difhhsize"}\NormalTok{,}\StringTok{"difwin1"}\NormalTok{,}\StringTok{"sumurban"}\NormalTok{,}\StringTok{"sumfemale"}\NormalTok{,}\StringTok{"sumyage"}\NormalTok{, }\StringTok{"sumysch"}\NormalTok{, }\StringTok{"nummarried"}\NormalTok{, }\StringTok{"sumtcons"}\NormalTok{, }\StringTok{"sumcons"}\NormalTok{, }\StringTok{"sumhhsize"}\NormalTok{, }\StringTok{"sumwin1"}\NormalTok{,}\StringTok{"Observations"}\NormalTok{))}

\NormalTok{Dyadic3 <-}\StringTok{ }\NormalTok{Dyadic }\OperatorTok
\StringTok{  }\KeywordTok{subset}\NormalTok{(inreg }\OperatorTok{==}\StringTok{ }\DecValTok{1} \OperatorTok{&}\StringTok{ }\NormalTok{frfamcl}\OperatorTok{==}\StringTok{ }\DecValTok{0}\NormalTok{)}\OperatorTok
\StringTok{  }\KeywordTok{mutate}\NormalTok{(}\DataTypeTok{Observations =} \DecValTok{0}\NormalTok{)}\OperatorTok
\StringTok{  }\KeywordTok{select}\NormalTok{(}\KeywordTok{c}\NormalTok{(}\StringTok{"samegroup"}\NormalTok{,}\StringTok{"difchoice1"}\NormalTok{,}\StringTok{"sumchoice1"}\NormalTok{,}\StringTok{"frfam"}\NormalTok{,}\StringTok{"friend2"}\NormalTok{,}\StringTok{"friendfamily"}\NormalTok{, }\StringTok{"family2"}\NormalTok{, }\StringTok{"friend1"}\NormalTok{, }\StringTok{"family1"}\NormalTok{,}\StringTok{"stranger"}\NormalTok{,}\StringTok{"frfamcl"}\NormalTok{, }\StringTok{"difurban"}\NormalTok{, }\StringTok{"diffemale"}\NormalTok{, }\StringTok{"difyage"}\NormalTok{, }\StringTok{"difysch"}\NormalTok{, }\StringTok{"difmarried"}\NormalTok{,}\StringTok{"diftcons"}\NormalTok{,}\StringTok{"difcons"}\NormalTok{, }\StringTok{"difhhsize"}\NormalTok{,}\StringTok{"difwin1"}\NormalTok{,}\StringTok{"sumurban"}\NormalTok{,}\StringTok{"sumfemale"}\NormalTok{,}\StringTok{"sumyage"}\NormalTok{, }\StringTok{"sumysch"}\NormalTok{, }\StringTok{"nummarried"}\NormalTok{, }\StringTok{"sumtcons"}\NormalTok{, }\StringTok{"sumcons"}\NormalTok{, }\StringTok{"sumhhsize"}\NormalTok{, }\StringTok{"sumwin1"}\NormalTok{,}\StringTok{"Observations"}\NormalTok{))}
\end{Highlighting}
\end{Shaded}

\begin{Shaded}
\begin{Highlighting}[]
\NormalTok{pervar<-}\KeywordTok{c}\NormalTok{(}\StringTok{"samegroup"}\NormalTok{,}\StringTok{"friend2"}\NormalTok{,}\StringTok{"friendfamily"}\NormalTok{,}\StringTok{"stranger"}\NormalTok{, }\StringTok{"frfam"}\NormalTok{,}\StringTok{"family2"}\NormalTok{, }\StringTok{"friend1"}\NormalTok{, }\StringTok{"family1"}\NormalTok{, }\StringTok{"difurban"}\NormalTok{,}\StringTok{"frfamcl"}\NormalTok{, }\StringTok{"diffemale"}\NormalTok{,}\StringTok{"difmarried"}\NormalTok{)}

\ControlFlowTok{for}\NormalTok{ (i }\ControlFlowTok{in}\NormalTok{ pervar) \{}
\NormalTok{Dyadic1[[i]] <-}\StringTok{ }\NormalTok{(}\KeywordTok{as.numeric}\NormalTok{(Dyadic1[[i]]))}\OperatorTok{*}\DecValTok{100}
\NormalTok{\}}


\NormalTok{pervar<-}\KeywordTok{c}\NormalTok{(}\StringTok{"samegroup"}\NormalTok{,}\StringTok{"friend2"}\NormalTok{,}\StringTok{"friendfamily"}\NormalTok{,}\StringTok{"stranger"}\NormalTok{, }\StringTok{"frfam"}\NormalTok{,}\StringTok{"family2"}\NormalTok{, }\StringTok{"friend1"}\NormalTok{, }\StringTok{"family1"}\NormalTok{, }\StringTok{"difurban"}\NormalTok{,}\StringTok{"frfamcl"}\NormalTok{, }\StringTok{"diffemale"}\NormalTok{,}\StringTok{"difmarried"}\NormalTok{)}

\ControlFlowTok{for}\NormalTok{ (i }\ControlFlowTok{in}\NormalTok{ pervar) \{}
\NormalTok{Dyadic2[[i]] <-}\StringTok{ }\NormalTok{(}\KeywordTok{as.numeric}\NormalTok{(Dyadic2[[i]]))}\OperatorTok{*}\DecValTok{100}
\NormalTok{\}}

\NormalTok{pervar<-}\KeywordTok{c}\NormalTok{(}\StringTok{"samegroup"}\NormalTok{,}\StringTok{"friend2"}\NormalTok{,}\StringTok{"friendfamily"}\NormalTok{,}\StringTok{"stranger"}\NormalTok{, }\StringTok{"frfam"}\NormalTok{,}\StringTok{"family2"}\NormalTok{, }\StringTok{"friend1"}\NormalTok{, }\StringTok{"family1"}\NormalTok{, }\StringTok{"difurban"}\NormalTok{,}\StringTok{"frfamcl"}\NormalTok{, }\StringTok{"diffemale"}\NormalTok{,}\StringTok{"difmarried"}\NormalTok{)}

\ControlFlowTok{for}\NormalTok{ (i }\ControlFlowTok{in}\NormalTok{ pervar) \{}
\NormalTok{Dyadic3[[i]] <-}\StringTok{ }\NormalTok{(}\KeywordTok{as.numeric}\NormalTok{(Dyadic3[[i]]))}\OperatorTok{*}\DecValTok{100}
\NormalTok{\}}
\end{Highlighting}
\end{Shaded}

\begin{Shaded}
\begin{Highlighting}[]
\KeywordTok{library}\NormalTok{(fBasics)}
\end{Highlighting}
\end{Shaded}

\begin{verbatim}
## Loading required package: timeDate
\end{verbatim}

\begin{verbatim}
## Loading required package: timeSeries
\end{verbatim}

\begin{Shaded}
\begin{Highlighting}[]
\NormalTok{sum1<-}\KeywordTok{basicStats}\NormalTok{(Dyadic1)[}\KeywordTok{c}\NormalTok{(}\StringTok{"Mean"}\NormalTok{, }\StringTok{"Stdev"}\NormalTok{),]}
\NormalTok{sum1<-}\KeywordTok{data.frame}\NormalTok{(}\KeywordTok{t}\NormalTok{(sum1))}
\NormalTok{sum2<-}\KeywordTok{basicStats}\NormalTok{(Dyadic2)[}\KeywordTok{c}\NormalTok{(}\StringTok{"Mean"}\NormalTok{),]}
\NormalTok{sum2<-}\KeywordTok{data.frame}\NormalTok{(}\KeywordTok{t}\NormalTok{(sum2))}
\NormalTok{sum3<-}\KeywordTok{basicStats}\NormalTok{(Dyadic3)[}\KeywordTok{c}\NormalTok{(}\StringTok{"Mean"}\NormalTok{),]}
\NormalTok{sum3<-}\KeywordTok{data.frame}\NormalTok{(}\KeywordTok{t}\NormalTok{(sum3))}
\end{Highlighting}
\end{Shaded}

\begin{Shaded}
\begin{Highlighting}[]
\NormalTok{table4<-}\KeywordTok{cbind}\NormalTok{(sum1,sum2,sum3)}
\NormalTok{table4[}\DecValTok{17}\NormalTok{,}\DecValTok{1}\NormalTok{]<-(table4[}\DecValTok{17}\NormalTok{,}\DecValTok{1}\NormalTok{])}\OperatorTok{/}\DecValTok{1000}
\NormalTok{table4[}\DecValTok{17}\NormalTok{,}\DecValTok{2}\NormalTok{]<-(table4[}\DecValTok{17}\NormalTok{,}\DecValTok{2}\NormalTok{])}\OperatorTok{/}\DecValTok{1000}
\NormalTok{table4[}\DecValTok{17}\NormalTok{,}\DecValTok{3}\NormalTok{]<-table4[}\DecValTok{17}\NormalTok{,}\DecValTok{3}\NormalTok{]}\OperatorTok{/}\DecValTok{1000}
\NormalTok{table4[}\DecValTok{17}\NormalTok{,}\DecValTok{4}\NormalTok{]<-table4[}\DecValTok{17}\NormalTok{,}\DecValTok{4}\NormalTok{]}\OperatorTok{/}\DecValTok{1000} 
\NormalTok{table4[}\DecValTok{26}\NormalTok{,}\DecValTok{1}\NormalTok{]<-table4[}\DecValTok{26}\NormalTok{,}\DecValTok{1}\NormalTok{]}\OperatorTok{/}\DecValTok{1000} 
\NormalTok{table4[}\DecValTok{26}\NormalTok{,}\DecValTok{2}\NormalTok{]<-table4[}\DecValTok{26}\NormalTok{,}\DecValTok{2}\NormalTok{]}\OperatorTok{/}\DecValTok{1000}
\NormalTok{table4[}\DecValTok{26}\NormalTok{,}\DecValTok{3}\NormalTok{]<-table4[}\DecValTok{26}\NormalTok{,}\DecValTok{3}\NormalTok{]}\OperatorTok{/}\DecValTok{1000}
\NormalTok{table4[}\DecValTok{26}\NormalTok{,}\DecValTok{4}\NormalTok{]<-table4[}\DecValTok{26}\NormalTok{,}\DecValTok{4}\NormalTok{]}\OperatorTok{/}\DecValTok{1000}

\NormalTok{is.num <-}\StringTok{ }\KeywordTok{sapply}\NormalTok{(table4, is.numeric)}
\NormalTok{table4[is.num] <-}\StringTok{ }\KeywordTok{lapply}\NormalTok{(table4[is.num], round, }\DecValTok{3}\NormalTok{)}
\end{Highlighting}
\end{Shaded}

\begin{Shaded}
\begin{Highlighting}[]
\CommentTok{#table4<-as.data.frame(lapply(table4, na.omit))}
\CommentTok{# table4<-transform(table4, Mean = as.numeric(Mean))}
\CommentTok{# table4<-transform(table4, Mean = as.numeric(Mean), Mean.1 = as.numeric(Mean.1), Mean.2 = as.numeric(Mean.2))}
\end{Highlighting}
\end{Shaded}

\begin{Shaded}
\begin{Highlighting}[]
\NormalTok{table4[}\DecValTok{1}\NormalTok{,}\DecValTok{1}\NormalTok{]<-}\KeywordTok{paste}\NormalTok{(table4[}\DecValTok{1}\NormalTok{,}\DecValTok{1}\NormalTok{],}\StringTok{"%"}\NormalTok{,}\DataTypeTok{sep =} \StringTok{""}\NormalTok{)}
\NormalTok{table4[}\DecValTok{1}\NormalTok{,}\DecValTok{3}\NormalTok{]<-}\KeywordTok{paste}\NormalTok{(table4[}\DecValTok{1}\NormalTok{,}\DecValTok{3}\NormalTok{],}\StringTok{"%"}\NormalTok{,}\DataTypeTok{sep =} \StringTok{""}\NormalTok{)}
\NormalTok{table4[}\DecValTok{1}\NormalTok{,}\DecValTok{4}\NormalTok{]<-}\KeywordTok{paste}\NormalTok{(table4[}\DecValTok{1}\NormalTok{,}\DecValTok{4}\NormalTok{],}\StringTok{"%"}\NormalTok{,}\DataTypeTok{sep =} \StringTok{""}\NormalTok{)}
\end{Highlighting}
\end{Shaded}

\begin{Shaded}
\begin{Highlighting}[]
\NormalTok{table4[}\DecValTok{4}\OperatorTok{:}\DecValTok{13}\NormalTok{,}\DecValTok{1}\NormalTok{]<-}\KeywordTok{paste}\NormalTok{(table4[}\DecValTok{4}\OperatorTok{:}\DecValTok{13}\NormalTok{,}\DecValTok{1}\NormalTok{],}\StringTok{"%"}\NormalTok{,}\DataTypeTok{sep =} \StringTok{""}\NormalTok{)}
\NormalTok{table4[}\DecValTok{4}\OperatorTok{:}\DecValTok{13}\NormalTok{,}\DecValTok{3}\NormalTok{]<-}\KeywordTok{paste}\NormalTok{(table4[}\DecValTok{4}\OperatorTok{:}\DecValTok{13}\NormalTok{,}\DecValTok{3}\NormalTok{],}\StringTok{"%"}\NormalTok{,}\DataTypeTok{sep =} \StringTok{""}\NormalTok{)}
\NormalTok{table4[}\DecValTok{4}\OperatorTok{:}\DecValTok{13}\NormalTok{,}\DecValTok{4}\NormalTok{]<-}\KeywordTok{paste}\NormalTok{(table4[}\DecValTok{4}\OperatorTok{:}\DecValTok{13}\NormalTok{,}\DecValTok{4}\NormalTok{],}\StringTok{"%"}\NormalTok{,}\DataTypeTok{sep =} \StringTok{""}\NormalTok{)}
\NormalTok{table4[}\DecValTok{16}\NormalTok{,}\DecValTok{1}\NormalTok{]<-}\KeywordTok{paste}\NormalTok{(table4[}\DecValTok{16}\NormalTok{,}\DecValTok{1}\NormalTok{],}\StringTok{"%"}\NormalTok{,}\DataTypeTok{sep =} \StringTok{""}\NormalTok{)}
\NormalTok{table4[}\DecValTok{16}\NormalTok{,}\DecValTok{3}\NormalTok{]<-}\KeywordTok{paste}\NormalTok{(table4[}\DecValTok{16}\NormalTok{,}\DecValTok{3}\NormalTok{],}\StringTok{"%"}\NormalTok{,}\DataTypeTok{sep =} \StringTok{""}\NormalTok{)}
\NormalTok{table4[}\DecValTok{16}\NormalTok{,}\DecValTok{4}\NormalTok{]<-}\KeywordTok{paste}\NormalTok{(table4[}\DecValTok{16}\NormalTok{,}\DecValTok{4}\NormalTok{],}\StringTok{"%"}\NormalTok{,}\DataTypeTok{sep =} \StringTok{""}\NormalTok{)}
\NormalTok{table4[}\DecValTok{1}\NormalTok{,}\DecValTok{2}\NormalTok{]<-}\StringTok{" "}
\NormalTok{table4[}\DecValTok{4}\OperatorTok{:}\DecValTok{13}\NormalTok{,}\DecValTok{2}\NormalTok{]<-}\StringTok{" "}
\NormalTok{table4[}\DecValTok{16}\NormalTok{,}\DecValTok{2}\NormalTok{]<-}\StringTok{" "}
\end{Highlighting}
\end{Shaded}

\begin{Shaded}
\begin{Highlighting}[]
\NormalTok{table4[}\DecValTok{30}\NormalTok{,]<-}\KeywordTok{c}\NormalTok{(}\StringTok{"86518"}\NormalTok{,}\StringTok{" "}\NormalTok{,}\StringTok{"4466"}\NormalTok{,}\StringTok{"82052"}\NormalTok{)}

\NormalTok{is.num <-}\StringTok{ }\KeywordTok{sapply}\NormalTok{(table4, is.numeric)}
\NormalTok{table4[is.num] <-}\StringTok{ }\KeywordTok{lapply}\NormalTok{(table4[is.num], round, }\DecValTok{3}\NormalTok{)}
\end{Highlighting}
\end{Shaded}

\begin{Shaded}
\begin{Highlighting}[]
\CommentTok{#position = "float_right" }
\KeywordTok{library}\NormalTok{(kableExtra)}
\end{Highlighting}
\end{Shaded}

\begin{verbatim}
## 
## Attaching package: 'kableExtra'
\end{verbatim}

\begin{verbatim}
## The following object is masked from 'package:dplyr':
## 
##     group_rows
\end{verbatim}

\begin{Shaded}
\begin{Highlighting}[]
\KeywordTok{library}\NormalTok{(params)}
\end{Highlighting}
\end{Shaded}

\begin{verbatim}
## Loading required package: whisker
\end{verbatim}

\begin{verbatim}
## 
## Attaching package: 'params'
\end{verbatim}

\begin{verbatim}
## The following object is masked from 'package:kableExtra':
## 
##     kable
\end{verbatim}

\begin{Shaded}
\begin{Highlighting}[]
\KeywordTok{library}\NormalTok{(knitr)}
\end{Highlighting}
\end{Shaded}

\begin{verbatim}
## 
## Attaching package: 'knitr'
\end{verbatim}

\begin{verbatim}
## The following object is masked from 'package:params':
## 
##     kable
\end{verbatim}

\begin{Shaded}
\begin{Highlighting}[]
\NormalTok{knitr}\OperatorTok{::}\KeywordTok{kable}\NormalTok{(table4,}\DataTypeTok{format =} \StringTok{"latex"}\NormalTok{,}\DataTypeTok{row.names =} \OtherTok{NA}\NormalTok{, }\DataTypeTok{col.names =} \KeywordTok{c}\NormalTok{(}\StringTok{"Mean/Prop"}\NormalTok{,}\StringTok{"s.d."}\NormalTok{,}\StringTok{"Mean/Prop"}\NormalTok{,}\StringTok{"s.d."}\NormalTok{), }\DataTypeTok{align =} \StringTok{'l'}\NormalTok{,}\DataTypeTok{escape =}\NormalTok{ F,}\DataTypeTok{booktabs =}\NormalTok{ F)}\OperatorTok
\StringTok{ }\KeywordTok{add_header_above}\NormalTok{(}\KeywordTok{c}\NormalTok{(}\StringTok{" "}\NormalTok{ =}\StringTok{ }\DecValTok{1}\NormalTok{, }\StringTok{"All dyads"}\NormalTok{ =}\StringTok{ }\DecValTok{2}\NormalTok{,}\StringTok{"Close family and friends"}\NormalTok{ =}\StringTok{ }\DecValTok{1}\NormalTok{, }\StringTok{"Other dyads"}\NormalTok{ =}\DecValTok{1}\NormalTok{))}\OperatorTok\StringTok{  }\KeywordTok{kable_styling}\NormalTok{(}\DataTypeTok{bootstrap_options =} \KeywordTok{c}\NormalTok{(}\StringTok{"striped"}\NormalTok{), }\DataTypeTok{full_width =}\NormalTok{ F, }\DataTypeTok{font_size =} \DecValTok{16}\NormalTok{)}\OperatorTok\KeywordTok{column_spec}\NormalTok{(}\DecValTok{1}\NormalTok{,}\DataTypeTok{border_right =}\NormalTok{ T)}\OperatorTok\KeywordTok{column_spec}\NormalTok{(}\DecValTok{2}\NormalTok{,}\DataTypeTok{italic =}\NormalTok{ T)}\OperatorTok\KeywordTok{add_header_above}\NormalTok{(}\KeywordTok{c}\NormalTok{(}\StringTok{"Table4: Experimental data"}\NormalTok{=}\DecValTok{5}\NormalTok{))}
\end{Highlighting}
\end{Shaded}

\begin{table}[H]
\centering\begingroup\fontsize{16}{18}\selectfont

\begin{tabular}{>{}l||>{\em}l|l|l|l}
\hline
\multicolumn{5}{c|}{Table4: Experimental data} \\
\cline{1-5}
\multicolumn{1}{c|}{ } & \multicolumn{2}{c|}{All dyads} & \multicolumn{1}{c|}{Close family and friends} & \multicolumn{1}{c}{Other dyads} \\
\cline{2-3} \cline{4-4} \cline{5-5}
  & Mean/Prop & s.d. & Mean/Prop & s.d.\\
\hline
samegroup & 9.21% &  & 29.467% & 8.107%\\
\hline
difchoice1 & 1.639 & 1.265 & 1.682 & 1.637\\
\hline
sumchoice1 & 7.179 & 2.123 & 7.056 & 7.186\\
\hline
frfam & 10.49% &  & 100% & 5.618%\\
\hline
friend2 & 2.427% &  & 29.019% & 0.98%\\
\hline
friendfamily & 0.176% &  & 2.015% & 0.076%\\
\hline
family2 & 0.453% &  & 5.464% & 0.18%\\
\hline
friend1 & 6.9% &  & 57.77% & 4.132%\\
\hline
family1 & 0.534% &  & 5.732% & 0.251%\\
\hline
stranger & 89.51% &  & 0% & 94.382%\\
\hline
frfamcl & 5.162% &  & 100% & 0%\\
\hline
difurban & 30.951% &  & 0% & 32.635%\\
\hline
diffemale & 20.537% &  & 16.57% & 20.753%\\
\hline
difyage & 12.404 & 9.682 & 11.481 & 12.454\\
\hline
difysch & 3.235 & 2.77 & 2.633 & 3.267\\
\hline
difmarried & 34.684% &  & 30.721% & 34.9%\\
\hline
diftcons & 232.84 & 227.244 & 226.585 & 233.181\\
\hline
difcons & 0.589 & 0.489 & 0.583 & 0.59\\
\hline
difhhsize & 3.111 & 2.907 & 2.803 & 3.128\\
\hline
difwin1 & 4.182 & 3.213 & 4.079 & 4.188\\
\hline
sumurban & 0.715 & 0.78 & 1.161 & 0.691\\
\hline
sumfemale & 1.75 & 0.482 & 1.765 & 1.749\\
\hline
sumyage & 83.673 & 16.012 & 84.456 & 83.631\\
\hline
sumysch & 7.352 & 4.512 & 6.71 & 7.387\\
\hline
nummarried & 1.55 & 0.592 & 1.595 & 1.548\\
\hline
sumtcons & 850.188 & 359.942 & 850.359 & 850.179\\
\hline
sumcons & 25.621 & 0.845 & 25.601 & 25.622\\
\hline
sumhhsize & 14.568 & 4.529 & 14.114 & 14.592\\
\hline
sumwin1 & 11.708 & 5.484 & 11.524 & 11.718\\
\hline
Observations & 86518 &  & 4466 & 82052\\
\hline
\end{tabular}
\endgroup{}
\end{table}


\end{document}
