\documentclass[]{elsarticle} %review=doublespace preprint=single 5p=2 column
%%% Begin My package additions %%%%%%%%%%%%%%%%%%%
\usepackage[hyphens]{url}



\usepackage{lineno} % add
\providecommand{\tightlist}{%
  \setlength{\itemsep}{0pt}\setlength{\parskip}{0pt}}

\bibliographystyle{elsarticle-harv}
\biboptions{sort&compress} % For natbib
\usepackage{graphicx}
\usepackage{booktabs} % book-quality tables
%%%%%%%%%%%%%%%% end my additions to header

\usepackage[T1]{fontenc}
\usepackage{lmodern}
\usepackage{amssymb,amsmath}
\usepackage{ifxetex,ifluatex}
\usepackage{fixltx2e} % provides \textsubscript
% use upquote if available, for straight quotes in verbatim environments
\IfFileExists{upquote.sty}{\usepackage{upquote}}{}
\ifnum 0\ifxetex 1\fi\ifluatex 1\fi=0 % if pdftex
  \usepackage[utf8]{inputenc}
\else % if luatex or xelatex
  \usepackage{fontspec}
  \ifxetex
    \usepackage{xltxtra,xunicode}
  \fi
  \defaultfontfeatures{Mapping=tex-text,Scale=MatchLowercase}
  \newcommand{\euro}{€}
\fi
% use microtype if available
\IfFileExists{microtype.sty}{\usepackage{microtype}}{}
\ifxetex
  \usepackage[setpagesize=false, % page size defined by xetex
              unicode=false, % unicode breaks when used with xetex
              xetex]{hyperref}
\else
  \usepackage[unicode=true]{hyperref}
\fi
\hypersetup{breaklinks=true,
            bookmarks=true,
            pdfauthor={},
            pdftitle={},
            colorlinks=false,
            urlcolor=blue,
            linkcolor=magenta,
            pdfborder={0 0 0}}
\urlstyle{same}  % don't use monospace font for urls

\setcounter{secnumdepth}{0}
% Pandoc toggle for numbering sections (defaults to be off)
\setcounter{secnumdepth}{0}
% Pandoc header
\setlength{\parindent}{2em}
\setlength{\parskip}{0.2em}



\begin{document}
\begin{frontmatter}

  \title{}
    \author[Smith College]{Jocelyn Jingjing Hu}
   \ead{jhu70@smith.edu} 
  
    \author[Smith College]{Richelle Ju}
   \ead{tju@smith.edu} 
  
    \author[Smith College]{Tiphanie Chow}
   \ead{tchow@smith.edu} 
  
    \author[Hampshire College]{Katherine Lim}
   \ead{kql15@hampshire.edu} 
  
      \address[Smith College]{Department of Economics, Pierce Hall, Northampton, MA, 01063}
  
  \begin{abstract}
  An important principle of economics is that people respond to
  incentives. For that reason, government and organizations implement
  incentives to encourage prosocial behavior. This paper looks at how
  monetary and nonmonetary incentives change people's decisions to act
  prosocially, and how these effects differ across gender. It will start
  by discussing the paper about the effect of monetary incentives on
  charity donations, and will then look at the literature on the drivers
  of prosocial behavior and the effectiveness of incentives. Lastly, we
  will tackle unanswered questions by suggesting a more comprehensive
  experimental design that addresses the gaps in the literature by
  considering a new population. \par
  \textbf{JEL Classification:} C91, C93, D90 \par
  \textbf{Keywords:} Individual Behavior, Laboratory Experiments, Field
  Experiment, Prosocial behavior
  \end{abstract}
  
 \end{frontmatter}

\section{1. Introduction}\label{introduction}

\section{2. Main paper}\label{main-paper}

\section{3. Reproduction of}\label{reproduction-of}

The following section will discuss the reproduction of all the relevant
figures, p-values and regression results to ``Click for Charity''
experiment presented in the paper. The discussed regression tables are
at the end of the paper, and the relevant regression for the mentioned
p-values can be found in the appendix. We weren't able to replicate the
results for ``Bike for Charity'' as the data was not available.

As in the, the analysis of the results will focus first on the good
cause and the bad cause cases separately in two figures below and then
finally analyze of all causes together by running an OLS regression. The
two figures below are substantially similar and don't differ graphically
from the original paper's.

Figure 1 below summarizes the effort of participants for the cause in
the public and private conditions, with and without incentives. The
first panel (A) defines the good cause by the participant's perception
of the cause, while the second panel (B) defines it by the consensus of
the Princeton students (Red Cross for the good cause). The figure shows
the average number of keypresses and their standard errors by payment
scheme and visibility of the participation. The quantitative results are
consistent across both panels. Firstly, in support of the Image
motivation hypothesis, we see that without incentives, participants
produce more effort in the public conditions. The difference was
significant at the 5\% and 1\% level for panel A and B respectively. The
introduction of monetary incentives does not increase the effort in the
public conditions, but does so in private condition. In the public
condition, there is a non-statistically significant decline
(p\textgreater{}0.3 for both panels) in effort. In the private
condition, there is a significant increase (p=0.048 and p=0.023 for
panel A and B respectively) in effort. These results support the
effectiveness hypothesis.

\% Table created by stargazer v.5.2.2 by Marek Hlavac, Harvard
University. E-mail: hlavac at fas.harvard.edu \% Date and time: Sun, May
05, 2019 - 12:53:50 AM

\begin{table}[!htbp] \centering 
  \caption{Table A1: Group-level analysis of defections: Dependent variable = proportion of members that default} 
  \label{} 
\scriptsize 
\begin{tabular}{@{\extracolsep{0.3pt}}lccc} 
\\[-1.8ex]\hline 
\hline \\[-1.8ex] 
\\[-1.8ex] & \multicolumn{3}{c}{Proportion of members that default} \\ 
\\[-1.8ex] & \textit{felm} & \multicolumn{2}{c}{\textit{OLS}} \\ 
 & Groups of 2 or 3 & All groups &  \\ 
\\[-1.8ex] & (1) & (2) & (3)\\ 
\hline \\[-1.8ex] 
 1:Density of close friends and family network within group & $-$0.067$^{***}$ & $-$0.034 & $-$0.155$^{**}$ \\ 
  & (0.021) & (0.029) & (0.066) \\ 
  2:Number of group members & 0.030 & 0.001 & $-$0.004 \\ 
  & (0.021) & (0.004) & (0.005) \\ 
  1x2 &  &  & 0.039$^{**}$ \\ 
  &  &  & (0.019) \\ 
  Average gamble choice & 0.012 & 0.012 & 0.011 \\ 
  & (0.011) & (0.009) & (0.009) \\ 
  Proportion of females & 0.050 & 0.032 & 0.031 \\ 
  & (0.048) & (0.037) & (0.037) \\ 
  Average age & $-$0.000 & $-$0.000 & $-$0.000 \\ 
  & (0.002) & (0.001) & (0.001) \\ 
  Proportion living in municipal centre & 0.082$^{**}$ & 0.031 & 0.021 \\ 
  & (0.036) & (0.031) & (0.031) \\ 
  Average years of education & $-$0.013$^{***}$ & $-$0.010$^{**}$ & $-$0.010$^{**}$ \\ 
  & (0.005) & (0.005) & (0.005) \\ 
  Proportion married & $-$0.047 & $-$0.001 & $-$0.004 \\ 
  & (0.048) & (0.035) & (0.035) \\ 
  Average log household consumption & 0.007 & $-$0.019 & $-$0.018 \\ 
  & (0.027) & (0.025) & (0.025) \\ 
  Average household size & 0.004 & 0.004 & 0.004 \\ 
  & (0.006) & (0.004) & (0.004) \\ 
  Constant & $-$0.124 & 0.212 & 0.218 \\ 
  & (0.340) & (0.312) & (0.311) \\ 
 Municipality dummies & no & no & yes \\ 
Observations & 251 & 526 & 526 \\ 
\hline \\[-1.8ex] 
\textit{Notes:} & \multicolumn{3}{l}{$^{***}$Significant at the 1 percent level.} \\ 
 & \multicolumn{3}{l}{$^{**}$Significant at the 5 percent level.} \\ 
 & \multicolumn{3}{l}{$^{*}$Significant at the 10 percent level.} \\ 
 & \multicolumn{3}{l}{Linear regression coefficients reported.} \\ 
 & \multicolumn{3}{l}{Standard errors (in parentheses)} \\ 
 & \multicolumn{3}{l}{adjusted to account of non-independence} \\ 
 & \multicolumn{3}{l}{within municipalities by clustering} \\ 
\end{tabular} 
\end{table}

\begin{center}\includegraphics[width=1\linewidth]{Final_paper_files/figure-latex/unnamed-chunk-4-1} \end{center}

\begin{table}[!h]
\centering\begingroup\fontsize{8}{10}\selectfont

\begin{tabular}{lrllrll}
\toprule
\multicolumn{1}{c}{Table 1: Experimental subjects} \\
\cmidrule(l{3pt}r{3pt}){1-1}
\multicolumn{1}{c}{ } & \multicolumn{3}{c}{Full Sample} & \multicolumn{3}{c}{Sample Analysed} \\
\cmidrule(l{3pt}r{3pt}){2-4} \cmidrule(l{3pt}r{3pt}){5-7}
  & Obs. & Mean/Prop & s.d. & Obs. & Mean/Prop & s.d.\\
\midrule
Female & 2420 & 87.52\% & 33.06 & 2321 & 87.2\% & 33.41\\
Age (years) & 2396 & 41.78 & 11.39 & 2321 & 41.72 & 11.37\\
Education (years) & 2397 & 3.7 & 3.12 & 2321 & 3.7 & 3.13\\
Household head & 2423 & 28.64\% &  & 2321 & 28.78\% & \\
Married & 2420 & 77.11\% &  & 2321 & 77.73\% & \\
\addlinespace
Lives in municipal centre & 2478 & 33.86\% & 47.33 & 2321 & 34.38\% & 47.51\\
Household consumption ('000 Pesos/month) 2 & 2478 & 433.64 & 254.9 & 2321 & 427.28 & 249.91\\
Log household consumption per month & 2478 & 12.82 &  & 2321 & 12.81 & \\
Household size & 2452 & 7.34 & 3.19 & 2321 & 7.27 & 3.13\\
No. of kin recognized in session & 2506 & 0.32 &  & 2321 & 0.32 & \\
\addlinespace
No. of friends recognized in session & 2506 & 2.39 & 2.57 & 2321 & 2.42 & 2.57\\
\bottomrule
\end{tabular}
\endgroup{}
\end{table}

\begin{table}[!h]

\caption{\label{tab:table 2}Gamble choices}
\centering
\fontsize{6}{8}\selectfont
\begin{tabular}{lllllll}
\toprule
Gamble Choice & Low payoff(yellow) & High payoff(blue) & Expected value & Standard Deviation & Risk aversion class & CRRA range\\
\midrule
Gamble1 & 3000 & 3000 & 3000 & 0 & Extreme & infinity to 7.49\\
Gamble2 & 2700 & 5700 & 4200 & 2121 & Severe & 7.49 to 1.73\\
Gamble3 & 2400 & 7200 & 4800 & 3394 & Intermediate & 1.73 to 0.81\\
Gamble4 & 1800 & 9000 & 5400 & 5091 & Moderate & 0.81 to 0.46\\
Gamble5 & 1000 & 11000 & 6000 & 7071 & Slight-neutral & 0.47 to 0.00\\
\addlinespace
Gamble6 & 0 & 12000 & 6000 & 8485 & Neutral-negative & 0 to -ve infinity\\
\bottomrule
\end{tabular}
\end{table}

\begin{table}[!h]
\centering\begingroup\fontsize{8}{10}\selectfont

\begin{tabular}{llllll}
\toprule
\multicolumn{6}{c}{Table3: Experimental data} \\
\cmidrule(l{3pt}r{3pt}){1-6}
\multicolumn{2}{c}{ } & \multicolumn{2}{c}{Full Sample} & \multicolumn{2}{c}{Sample Analysed} \\
\cmidrule(l{3pt}r{3pt}){3-4} \cmidrule(l{3pt}r{3pt}){5-6}
Variables &   & Mean/Prop & s.d. & Mean/Prop & s.d.\\
\midrule
\addlinespace[0.3em]
\multicolumn{6}{l}{\textbf{Gamble choice 1st round}}\\
\hspace{1em}Gamble 1(safe) &  & 8.739\% &  & 8.746\% & \\
\hspace{1em}Gamble 2 &  & 17.757\% &  & 17.665\% & \\
\hspace{1em}Gamble 3 &  & 18.196\% &  & 18.311\% & \\
\hspace{1em}Gamble 4 &  & 29.29\% &  & 29.168\% & \\
\hspace{1em}Gamble 5 &  & 11.253\% &  & 11.116\% & \\
\hspace{1em}Gamble 6(riskiest) &  & 14.765\% &  & 14.994\% & \\
Won gamble in 1st round &  & 54.7\% &  & 54.5\% & \\
Winings 1st round ('000 Pesos) &  & 5.842 & 3.832 & 5.835 & 3.838\\
Joined a group &  & 86.2\% &  & 86.9\% & \\
Number of co-group members &  & 4.195 & 5.776 & 3.687 & 3.899\\
\addlinespace[0.3em]
\multicolumn{6}{l}{\textbf{Gamble choice 2nd round}}\\
\hspace{1em}Gamble 1(safe) &  & 6.028\% &  & 5.991\% & \\
\hspace{1em}Gamble 2 &  & 12.854\% &  & 12.759\% & \\
\hspace{1em}Gamble 3 &  & 17.685\% &  & 17.759\% & \\
\hspace{1em}Gamble 4 &  & 28.942\% &  & 28.75\% & \\
\hspace{1em}Gamble 5 &  & 17.206\% &  & 17.328\% & \\
\hspace{1em}Gamble 6(riskiest) &  & 17.285\% &  & 17.414\% & \\
Won gamble in 2nd round &  & 57.7\% &  & 57.7\% & \\
Reneged having won gamble &  & 6.3\% &  & 6.4\% & \\
Reneged having lost gamble &  & 1.8\% &  & 1.8\% & \\
Winnings 2nd round ('000 Pesos) &  & 6.134 & 4.046 & 6.133 & 4.052\\
Observations &  & 2506 &  & 2321 & \\
\bottomrule
\end{tabular}
\endgroup{}
\end{table}

\begin{table}[!h]
\centering\begingroup\fontsize{6}{8}\selectfont

\begin{tabular}{lllll}
\toprule
\multicolumn{5}{c}{Table4: Experimental data} \\
\cmidrule(l{3pt}r{3pt}){1-5}
\multicolumn{1}{c}{ } & \multicolumn{2}{c}{All dyads} & \multicolumn{1}{c}{Close family and friends} & \multicolumn{1}{c}{Other dyads} \\
\cmidrule(l{3pt}r{3pt}){2-3} \cmidrule(l{3pt}r{3pt}){4-4} \cmidrule(l{3pt}r{3pt}){5-5}
  & Mean/Prop & s.d. & Mean/Prop & s.d.\\
\midrule
Joined same group in round 2(\%) & 9.21\% &  & 29.467\% & 8.107\%\\
Difference in gamble choice(round1) & 1.639 & 1.265 & 1.682 & 1.637\\
Sum of gamble choices (round 1) & 7.179 & 2.123 & 7.056 & 7.186\\
Friends and family: One or both recognized friendship or kinship(\%) & 10.49\% &  & 100\% & 5.618\%\\
Both recognized friendship(\%) & 2.427\% &  & 29.019\% & 0.98\%\\
\addlinespace
Both recognized kinship(\%) & 0.176\% &  & 2.015\% & 0.076\%\\
One recognized friendship, other kinship(\%) & 0.453\% &  & 5.464\% & 0.18\%\\
One recognized friendship(\%) & 6.9\% &  & 57.77\% & 4.132\%\\
One recognized kinship(\%) & 0.534\% &  & 5.732\% & 0.251\%\\
Stranger(\%) & 89.51\% &  & 0\% & 94.382\%\\
\addlinespace
Geographically proximate friends and family(\%) & 5.162\% &  & 100\% & 0\%\\
One lives in the municipal centre, one not(\%) & 30.951\% &  & 0\% & 32.635\%\\
Different genders(\%) & 20.537\% &  & 16.57\% & 20.753\%\\
Difference in age(years) & 12.404 & 9.682 & 11.481 & 12.454\\
Difference in education (years) & 3.235 & 2.77 & 2.633 & 3.267\\
\addlinespace
Difference in marital status(\%) & 34.684\% &  & 30.721\% & 34.9\%\\
Difference in household consumption ('000s Pesos/month) & 232.84 & 227.244 & 226.585 & 233.181\\
Difference in log household consumption per month & 0.589 & 0.489 & 0.583 & 0.59\\
Difference in household size & 3.111 & 2.907 & 2.803 & 3.128\\
Difference in round 1 winnings ('000 Pesos) & 4.182 & 3.213 & 4.079 & 4.188\\
\addlinespace
Number who live in the municipal centre & 0.715 & 0.78 & 1.161 & 0.691\\
Number of females & 1.75 & 0.482 & 1.765 & 1.749\\
Sum of ages (years) & 83.673 & 16.012 & 84.456 & 83.631\\
Sum of education (years) & 7.352 & 4.512 & 6.71 & 7.387\\
Number married & 1.55 & 0.592 & 1.595 & 1.548\\
\addlinespace
Sum of household consumption ('000s Pesos/month) & 850.188 & 359.942 & 850.359 & 850.179\\
Sum of log household consumption per month & 25.621 & 0.845 & 25.601 & 25.622\\
Sum of household sizes & 14.568 & 4.529 & 14.114 & 14.592\\
Sum of round 1 winnings ('000s Pesos) & 11.708 & 5.484 & 11.524 & 11.718\\
obs & 86518 &  & 4466 & 82052\\
\bottomrule
\end{tabular}
\endgroup{}
\end{table}

The same analysis is replicated below for the bad cause. We observe that
without incentives, there is no significant difference
(p\textgreater{}0.44 for both panels) between the effort produced in the
public and private conditions. When given monetary incentives, we
observe the same results as for the good cause. In the public condition,
there is non-significant effect on effort (p\textgreater{}0.6 for both
panels). In the private condition, we observe an increase in effort, but
this difference in effectiveness across the two different visibilities
is significant for panel A (p=0.07) but not for panel B (p=0.32).

To analyze the two causes together, the following OLS regression tables
(Table 1 and 2) analyze how the private/public conditions affect the
number of pressed key pairs, which reflects the effort produced. The
first table controls for the perceived identification of others, and the
second controls for American Red Cross and NRA. The first column of both
tables show that monetary incentives increase effort in the private
conditions, which doesn't hold in the public condition as column 2
shows. The two tables also show the importance of the cause, especially
in the public conditions. The third columns, which combines the public
and private conditions, reinforces the previous stated results about the
effect of monetary incentives and the nature of the cause on the level
of effort. Overall, the coefficients of our table and their statistical
significance match the paper's results while our t-values differ
slightly from those in the paper. Two coefficients had different
significance levels in our results- one was significant at the 1\% level
in our table while it was significant at the 5\% level in the paper's
table, and the opposite was true for the other. We believe this change
is due that the p-values are close to the cutoff between the two
significance levels, so the difference might be due to the use of a
difference statistical software.

\section{4. Related literature}\label{related-literature}

\section{5. Experiment proposal}\label{experiment-proposal}

\section{Appendix}\label{appendix}

\section{References}\label{references}

\end{document}


